
\documentclass{article}
\usepackage{geometry}
\usepackage{fancyhdr}
\usepackage{lastpage}
\usepackage{amsmath}
\usepackage{textgreek}

\geometry{
a4paper,
total={170mm,257mm},
left=20mm,
top=20mm,
}

\pagestyle{fancy}
\fancyhf{}
\fancyhead[RE,LO]{Name:}
\fancyfoot[RE,LO]{Gesamtpunktzahl: 7P}
\fancyfoot[LE,RO]{Seite \thepage/\pageref{LastPage}}

\begin{document}

Fachbereich MND / WS 2021


\part*{Test 1 - Elektromagnetismus (7P)}

Die folgenden Aufgaben behandeln eine Luftspule mit der Windungszahl $N=170$, dem Radius $r=0{,}02$m bzw. $r=2*10^{ -2 }$m und der Länge $l=0{,}14$m.

\subsection*{Berechnen Sie die Induktivität der Spule (3P)}


\begin{gather}
A=2 * pi * r ^ 2=2{,}513*10^{-3}\text{m}^2 \\
L=N ^ 2 * mu_0 * A / l=6{,}52*10^{-3}\text{H}
\end{gather}

\subsection*{Bestimmen Sie den magnetischen Widerstand $R_{m}$ der Spule (4P)}


\begin{gather}
R_{m}=l / (mu_0 * A)=4{,}433*10^{6}\Omega
\end{gather}




\end{document}