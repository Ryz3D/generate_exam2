
\documentclass{article}
\usepackage{geometry}
\usepackage{fancyhdr}
\usepackage{lastpage}
\usepackage{amsmath}
\usepackage{textgreek}

\geometry{
a4paper,
total={170mm,257mm},
left=20mm,
top=20mm,
}

\pagestyle{fancy}
\fancyhf{}
\fancyhead[RE,LO]{Name:}
\fancyfoot[RE,LO]{Gesamtpunktzahl: 27P}
\fancyfoot[LE,RO]{Seite \thepage/\pageref{LastPage}}

\begin{document}

Fachbereich MND / WS 2021

\begin{table}[]
\begin{tabular}{ lll }
7 & 13 & 7
\end{tabular}
\end{table}


\part*{Test 1 - Elektromagnetismus (7P)}

Die folgenden Aufgaben behandeln eine Luftspule mit der Windungszahl $N=170$, dem Radius $r=0{,}02$m bzw. $r=2*10^{ -2 }$m und der Länge $l=0{,}2$m.

\subsection*{Berechnen Sie die Induktivität der Spule (3P)}


\vspace{\baselineskip}\vspace{\baselineskip}\vspace{\baselineskip}

\subsection*{Bestimmen Sie den magnetischen Widerstand $R_{m}$ der Spule (4P)}


\vspace{\baselineskip}



\part*{Test 1 - Elektromagnetismus (13P)}

Die folgenden Aufgaben behandeln eine Luftspule mit der Windungszahl $N=110$, dem Radius $r=0{,}02$m bzw. $r=2*10^{ -2 }$m und der Länge $l=0{,}16$m.

\subsection*{Berechnen Sie die Induktivität der Spule (7P)}


\vspace{\baselineskip}\vspace{\baselineskip}\vspace{\baselineskip}

\subsection*{Bestimmen Sie den magnetischen Widerstand $R_{m}$ der Spule (6P)}


\vspace{\baselineskip}



\part*{Test 1 - Elektromagnetismus (7P)}

Die folgenden Aufgaben behandeln eine Luftspule mit der Windungszahl $N=120$, dem Radius $r=0{,}02$m bzw. $r=2*10^{ -2 }$m und der Länge $l=0{,}13$m.

\subsection*{Berechnen Sie die Induktivität der Spule (3P)}


\vspace{\baselineskip}\vspace{\baselineskip}\vspace{\baselineskip}

\subsection*{Bestimmen Sie den magnetischen Widerstand $R_{m}$ der Spule (4P)}


\vspace{\baselineskip}




\end{document}